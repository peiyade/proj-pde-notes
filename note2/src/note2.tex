\documentclass[../../main.tex]{subfiles}
\addbibresource{\subfix{../note2.bib}}

\begin{document}

\chapter{Sobolev 空间速查}

\section{基本定义}
设 $1 \leq p \leq \infty$,Sobolev 空间
\[
  W^{k,p}(\Omega) = \left\{ u \in L^p(\Omega) \mid D^\alpha u \in L^p(\Omega),\ |\alpha| \leq k \right\}
\]
配以半范数
$|u|_{W^{k,p}(\Omega)} = \left( \sum_{|\alpha| = k} \|D^\alpha u\|_{L^p(\Omega)}^p \right)^{1/p}$。

\section{紧嵌入}
\begin{theorem}[Rellich--Kondrachov]
  令 $\Omega \subset \mathbb{R}^n$ 为有界 Lipschitz 区域,$1 \leq p < p^\ast < \infty$。若
  $k p < n$ 且 $p^\ast = \frac{np}{n - kp}$,则嵌入
  \[
    W^{k,p}(\Omega) \hookrightarrow L^{q}(\Omega)
  \]
  对所有 $1 \leq q < p^\ast$ 紧致。
\end{theorem}

\section{参考资料}
经典教材~\cite{gilbarg2001} 第 7 章给出了完整证明以及边界正则性的细节。

\printbibliography[heading=subbibliography]

\end{document}
